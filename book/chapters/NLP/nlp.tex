\chapterauthor{Christfried H. Focke}{AppFolio}
\chapterauthor{Rob M. Sylvester}{Reltio}

\chapter{Natural Language Processing}

In this chapter we introduce modern NLP libraries, techniques and their applications.
This chapter will focus on deep learning methods and less on computational linguistics that require nuanced knowledge of linguistics.
We explore what it means to represent words and sequences of words with rich numeric representations that are better-suited toward modern computational tasks.
We aim to capture some of these modern fine-tuned representations that are specially catered toward a semantic lexicon for medical language.
After finishing this chapter, one should have a general knowledge of the building blocks of state-of-the-art NLP models, and an understanding of how these
pieces fit together into a model such as GPT4\cite{openai2023gpt4}. Most importantly, we will present examples and patterns that help clinicians and researchers
use them in the field of epilepsy.
\\ 
\noindent This includes:
\begin{itemize}
\item An introduction to fundamental NLP concepts
\item A review of Deep Learning (DL) for NLP
\item A Discussion of embeddings and medical text representations
\item General use case patterns for NLP techniques in epilepsy
\item An example walkthrough using financial impact analysis data
\item A discussion of modern generative NLP models and their use in epilepsy
\end{itemize}

\section{Introduction to Natural Language Processing}

Natural language processing (NLP) is a field of computer science that deals with the extraction, processing, and understanding of human language.
It is known as the field of computer linguistics, and is a subfield of artificial intelligence.
Common NLP tasks include sentence segmentation, tokenization, part-of-speech tagging, named-entity recognition, parsing, question answering, summarization and classification.

How can we teach a computer to perform these tasks?
The first challenge is that computers, at their core, only understand numbers.
So first we need to think about how words, or more generally text, can be represented with numbers such that we can perform calculations on them.
The numbers should allow the computer to assign meaning to words and their context with other words around it.

There is no perfect way to represent language in this numeric fashion. The best we can do is to capture representations that we feel capture the \textit{syntactic} features of text, as well as \textit{semantic} features of text.
Syntactic information refers to grammatical constructs and the way that words are put together in a language. Semantic information refers to the meaning of this body of text. Such distinctions are fundamental concepts in computer science
and date back to the very foundations of information theory. \cite{shannon48}
Much of this chapter, as well as much of the current work in the field of computational linguistics is centered around finding better and better representations that capture both types of this information.

How do we begin to choose these representations?
Do we choose to represent words, or perhaps individual letters?
Do we choose all the words?
What do we do with punctuation and numbers?

\subsection{Vocabularies and Words}
\label{vocabularies_and_words}
In NLP, the set of unique words in a corpus is called a \textit{vocabulary}. While \textit{the} is the most common word in the English language, it is only one entry in a very large vocabulary table. A rare word like \textit{hemispherectomy} is also one
entry in this table. Both \textit{apple} and \textit{apples} might be separate entries in our vocabulary table. Generally the first step in many NLP systems is defining this vocabulary and the rules behind which words are in this table and which words are not.

The process of splitting text into separate smaller units, in this case words, is known as \textit{tokenization}.
When we run many NLP tasks, we almost always preprocess our text by putting it into one of these tokenizers.
The output tokens, in this case words, are the indexes in our lookup table to different numeric representations that can be understood by a machine.
There are a number of great tokenization libraries that are open source, and you should always use them rather than try to implement this yourself.

\begin{python}
  # Example 1: Spacy Tokenization Using Medical Words

  import spacy
  sample = "After a temporal lobe resection, the " \
           "atonic and clonic seizure frequency " \
           "fell by 50%."

  nlp = spacy.load("en_core_web_sm")
  english_vocab = set(nlp.vocab.strings)
  spacy_document = nlp(sample)

  for token in spacy_document:
      found = token.text in english_vocab

      print("{}: {}".format(token.text, found))
\end{python}

The discrete numeric entities created in Example 1 are the indexes for our representations.
The tokenizer system did much of the magic under the hood, lowercasing and standardizing text and creating special tokens for punctuation and numbers.

When we build a vocabulary $V$, we define a preset vocabulary size $|V|$. There is a tradeoff in your choice of $|V|$.
If you choose a number that is too small, then your system will only recognize a few words and lose a lot of important task-specific vocaublary.
However, if you include the entire english language, your system will be slow, expensive, and not perform as well.
Commonly this number is set around 20,000 - 40,000 words in English, though it can be much larger.

Often a vocabulary will include special tokens that make our lives easier when working with NLP tasks.
Common special tokens include:
\begin{itemize}
  \item \pythoninline{[EOS]} - An end of sentence/input marker.
  \item \pythoninline{[PAD]} - Special inputs to ignore, usually following an EOS marker.
  \item \pythoninline{[SEP]} - A separator, which can be used in inputs that contain multiple sentences or samples.
  \item \pythoninline{[OOV]} - Out of vocabulary, also commonly represented as [UNK] (unknown), which is used when we come across a word that does not exist in our vocabulary.
\end{itemize}

The out-of-vocabulary token is of particular importance in medicine. If we load a vocabulary with 30,000 words, often these are roughly the most common 30,000 words in the English language. We will still have a few [OOV] tokens
for rare words. However, just because a general system chooses its vocabulary based on word frequency does not mean that this is the best choice of vocabulary for the task at hand. Medical corpora have a very different word frequency
distribution than non-medical corpora, particulary because of nuanced vocabulary that is absolutely crucial.

%TODO - add some epilepsy-themed example that throws away all the epilepsy words%

In Example 1 above, notice the importance of the words that have been thrown away. If this sentence were to be loaded into a machine as-is, we might be losing far too much information because we have to nearly all of the relevant terminology with OOV tokens.
We will address this issue in the ensuing sections. For now, suffice it to say that we should make sure our vocabularies are catered toward our task at hand or are built in such a way that they can still extract signal from these tokens. Furthermore, we should always sanity check our sample token outputs against our raw text inputs.

\subsection{Word Representations}

Given that we have defined a vocabulary $V$ of words, these are the keys to our lookup table of numeric representations. But what do we store in that table as the value?

A simple way to achieve this representation for $V$ is to assign each word a unique integer $i$.
The word is then represented as a vector $w$ of length $|V|$ with all zeros and a one\footnote{This is also known as a one-hot encoding}. at index $i$.

\begin{python}
  have = [1, 0, 0, 0, 0, 0, ... 0]
  a    = [0, 1, 0, 0, 0, 0, ... 0]
  good = [0, 0, 1, 0, 0, 0, ... 0]
  day  = [0, 0, 0, 1, 0, 0, ... 0]
  ...
\end{python}

We could now represent a sentence as the sum of the word vectors $S = \sum_j w_j$. This representation is suitable as input for any classification algorithm (such as logistic regression or a decision tree), to make a prediction about our target variable, e.g. whether the sentence is relevant to our epilepsy task.
This simple representation fulfills our requirements but comes with some drawbacks:
\begin{itemize}
    \item We implicitly assume that each word in the sentence is equally important.
    \item Each word is equally similar to every other word (e.g. by taking the Euclidean distance between word vectors).
    \item The representation is invariant to a reordering of the words.
\end{itemize}
The first point is a problem, because as we add more word vectors together, the sentence representation will converge to the global average and drown out any signal relevant to the specific sentence at hand.
To address this we can instead write a weighted sum $S = \sum_j \lambda_j w_j$, where each word vector is weighted by its importance $\lambda_j$. There are statistical methods we can use to compute an importance value (e.g. TF-IDF).
For example a word like \textit{the} is very common and appears in many different contexts.
It is unlikely that there is a lot of signal we can extract from it.
On the other hand, a word like \textit{epilepsy} will be much more rare and specific to a given task, and we would like to raise its importance.
This has the net effect of allowing us to find good representations for larger word sequences.
To address the second point, we will give a brief introduction into Deep Learning (DL) and the extremely powerful concept of \textit{Embeddings}, one that is not only relevant in NLP, but in the world of AI in general.

\section{Deep Learning}

Neural networks have revolutionized the field of NLP. The field has seen three main architectures emerge over time: feedforward deep neural networks (DNNs), recurrent neural networks (RNNs), and transformers.

\subsection{Deep Neural Networks}
DNNs are the simplest form of neural networks and are characterized by their feedforward structure, where information flows only in one direction, from input to output.
DNNs were initially used in NLP for tasks such as sentiment analysis, where the goal was to predict a binary class label (e.g., positive or negative) given an input text.
The key advantage of DNNs was their ability to automatically learn complex representations of the input data using multiple hidden layers, allowing for the capture of high-level features that could not be easily hand-engineered.

However, DNNs have several limitations in NLP.
For example, they do not effectively capture the sequential structure of text data. To overcome this, RNNs were introduced.
RNNs are a type of neural network that have loops in their structure, allowing information to flow both forward and backward through the network.
This enables RNNs to model sequential data such as time series or text, by capturing the dependencies between consecutive elements.

Despite the effectiveness of RNNs, they suffer from the vanishing gradient problem.
This occurs when the gradients used to update the network's parameters become extremely small, making it difficult to train the network effectively.
To solve this, the attention mechanism was introduced in the transformer architecture.
Transformers are a type of neural network that is based on self-attention, which allows the network to attend to different parts of the input sequence in parallel.
This allows transformers to effectively capture long-term dependencies in sequential data, without the limitations of the vanishing gradient problem.

Transformers have proven to be highly effective in NLP, and have become the state-of-the-art architecture for many tasks, including machine translation, text classification, and question answering.
This is largely due to the ability of transformers to effectively capture the relationships between different parts of the input sequence, which is critical for NLP tasks where understanding the context and relationships between words is important.


\subsection{Transformers}
There have been significant advancements in ML due to a variety of factors, such as the availability of larger and larger datasets, as well as ever increasing computational power at lower and lower prices, better software tools, libraries and model architectures.
Especially the advent of deep learning (DL), fueled by the emergence of GPUs from about 2012, enabled scaling to many orders of magnitude larger datasets and models.
The common wisdom before DL was that as the number of parameters in a model increases beyond a certain threshold, it tends to overfit the training data, leading to poor generalization when deployed in the real world.
Deep Learning based models empirically do not seem to exhibit this behavior, and instead tend gain performance as they get bigger, albeit more slowly.

Until recently, building strong NLP systems required deep understanding of language and its structure, as well as large amounts of data, even when relying on pre-trained word embeddings - the resulting systems were often still brittle.

In recent years some surprisingly powerful architectures and pre-training regimes have emerged that build on the concepts we have discussed here, most notably under the name of \textit{Transformers\footnote{BERT, GPT}}, that address the last bullet point of our representation's shortcomings.

The details of the inner workings are beyond the scope of this tutorial, for now it suffices to know that they are able to learn robust sentence embeddings with a fine grained understanding of syntactic structure, semantics, and general knowledge of the world.
In fact, they can be considered near the level of a human that just knows the English language (or other languages), along with a broad array of factual knowledge about the world (think Wikipedia).
Just like a human, they can be taught specialty knowledge that is relevant to a given task, in ML we would say we \textit{finetune} them.
% Where these models to date generally fall short is in terms of logical inference.
This paradigm shift has led to a step function improvement in terms of performance and sample efficiency in a wide variety of NLP tasks where Transformers are the defacto state of the art.

In the following sections, we will discuss a broad set of problem statements Transformers are suitable for, along with a set of techniques a practitioner can employ to arrive at robust solutions, even with limited data.

\subsubsection{Pre-training}
Modern DL-based systems, such as the aforementioned Transformer architectures, are pre-trained on gigantic datasets\footnote{BooksCorpus (800M words, Wikipedia 2,500M words)}\cite{bertpaper} and can be downloaded for free\footnote{Huggingface}.
Starting with a model pre-trained on a broad set of topics significantly reduces the amount of task specific training data required to achieve a given performance.
Furthermore, it may be helpful to start with a model that is pre-trained on more domain-specific datasets such as \texttt{BioBERT\cite{DBLP:journals/corr/abs-1901-08746}} or \texttt{Bio\textunderscore ClinicalBERT\cite{clinicalbert}}, or pre-train your model from scratch.


\section{Embeddings}
\label{embeddings}
An embedding is a representation of a vector space that captures the similarity between the entities we are encoding, where entities can be words, images, e-commerce products, movies, houses, and many other data modalities.
Scientific progress in the field of ML is often directly about finding algorithms that can learn high-quality robust embeddings in an efficient and scalable way. The final performance of a given architecture is often largely determined by the quality of the embeddings it learns.

To illustrate this concept, we will be talking briefly about one of the earliest algorithms that has been used to learn embeddings for words: \textit{word2vec}, a technique pioneered by a team led by Thomas Mikolov.\cite{https://doi.org/10.48550/arxiv.1301.3781}
The idea behind \textit{word2vec} is to set a word into context with the words surrounding it. As it turns out, this idea that similar words appear in similar contexts, is an extremely powerful signal for achieving quality word embeddings. In NLP, the study of techniques that determine the probability of a given sequence of words is known as language modeling.

For example the word \textit{epilepsy} may appear in the context of words like \textit{symptom}, \textit{medication}, or \textit{episode} and vice versa.
However, we would not expect that \textit{epilepsy} would appear in the context of words like \textit{volleyball}.

We then translate this idea into a training task:
Given the sum of the one-hot encoded word vectors of the words surrounding the target word, map it to the one-hot encoded vector of the target word.

This is a task that can be solved by a neural network with the following setup:
Add a layer $l_1$ that maps from the input dimension (of size $|V|$) to an intermediate dimension of size $h$ and then another layer $l_2$ that maps from the intermediate dimension to the output dimension (of size $|V|$).
Here $h$ is what is called embedding dimension and $h << |V|$.
\begin{figure}
  \includegraphics[width=\linewidth]{chapters/NLP/figures/word2vec.png}
  \caption{Word2vec}
  \label{fig:word2vec}
\end{figure}
While $h$ is generally a hyperparameter that can be optimized at a later stage, a reasonable choice is to use a value according to the rule of thumb:
\begin{equation}
  h \approx 4\sqrt[\leftroot{3} \uproot{3} 4]{|V|}
\end{equation}
For a vocabulary size $|V| = 30000$ this comes to a value of about $50$.
During training, the network will learn to encode the original sparse indicator vectors in a way that preserves the maximum amount of information necessary to predict the target word, while being forced to squeeze the information through the low dimensional bottleneck.
An interesting side effect of this approach is that we can use part of the network to encode a given word in $V$ by using for example the inverse of $l_2$ as a lookup table.
The vectors that we obtain from this lookup table are called word embeddings and they have some very useful properties:
\begin{itemize}
    \item Similar words tend to be close together (e.g. in terms of Euclidean distance).
    \item Averaging word embeddings of a sentence will give us a more robust representation than our naive approach.
    \item We can perform some arithmetic, such as adding and subtracting the embedding vectors to traverse the space.
\end{itemize}
\begin{figure}
  \includegraphics[width=\linewidth]{chapters/NLP/figures/king-man+woman.png}
  \caption{Word2vec}
  \label{fig:kingmanwoman}
\end{figure}
These properties also imply that the representions for two sentences will be close together if they are semantically similar, which will make the job of, say, a downstream classifier much more straightforward - it only has to slice the embedding space.
The hard part of understanding the meaning of the sentence is already done.
Since the training data can be generated from just raw text, it is sufficient to train the embeddings once on a very large corpus\footnote{GloVe} and then reuse them, in effect giving us a headstart when building task-specific models.
Instead of also having to learn what words mean, and some of the syntactic rules behind written language, we can get a head start on this and focus our computational resources on adapting this knowledge to a specific task.

Using these pre-trained text representations is often critical. Without doing so, the majority of what a machine learns is the basic syntax and semantics of language. By using these pre-trained word embeddings, we allow our models to learn to focus more on the downstream task. By simplifying the learning task with these pre-trained embeddings, we need to use far fewer labeled examples than we would training from scratch.

This underlying assumption that similar neighboring words share a semantic similarity to center words is powerful, and at the heart of many modern training objectives for embeddings that are used in many downstream ML systems.
We do not inherently inject a great deal of model bias, but rather rely on having a large enough training set that the word vectors will become meaningful representations with respect to one another.
A corpus such as Common Crawl\footnote{Common Crawl Link}, which contains roughly 840 billion tokens of over two million unique words, provides such scale.
There are other corpora used in these unsupervised processes, one such common candidate being a collection of all Wikipedia articles.

In practice, one training objective can be to use the distributed representation of a set of context words to guess a center word. This is known as the continuous bag of words (CBOW) model.

Consider a training document describing patients with epilepsy, where we are attempting to train a model with a predefined vocabulary $V$ of the top For a vocabulary $V$, where we have chosen the top $|V|$ most common words in English to learn.
The document being chosen in this training iteration may contain the following excerpt:

\begin{verbatim}
  ... recommended administering Lamictal to treat their
  focal seizures, the patient continued experiencing
  symptoms though reported a 50% reduction in ...
  \end{verbatim}

In training our word embeddings, we will randomly mask one of the words, in this case \textit{patient}.
Our model makes use of an additional parameter, a context window size $c$, that helps to provide necessary context.
If we choose $c = 8$, our training sample would become

\begin{verbatim}
  ... lamictal to treat their focal seizures, the [MASK]
  continued experiencing symptoms though reported a 50 % ...
  \end{verbatim}

For a given center word at index $i$, we consider previous words at indexes $i-1$, $i-2$, ..., $i-c$, and all subsequent words at indexes $i+1$, $i+2$, ..., $i+c$.
In our word sequence $S$, we may then choose to sum the word vectors of each of these context words.
Our context input can be written as:
\begin{equation}
  \sum_{ \substack {j=-c \\ j \neq 0}}^c w_{i+j}
\end{equation}

Our task is then to predict a probability distribution over the size of our vocabulary $|V|$, where the target is a vector of all 0's with a 1 at the index in the vocabulary for the word \textit{patient}.

There are a few additional takeaways from this example.
Note that we will address many of these in the sections to follow.
\begin{itemize}
  \item The training process will likely consider this text many times, and mask out different words in each iteration.
  \item Our processed sentence is slightly modified from our original input.
  Notice that the number and percent symbol are split, and that Lamictal is lowercased.
  \item The word \textit{patient} can also be an adjective.
  Our \textit{patient} embedding will also capture this meaning in our model, dependent upon how often it is used with that meaning.
  \item If our context window is too small, we lose information.
  If it is too large, our signal becomes fuzzy as we converge to the global mean vector.
\end{itemize}

What if we choose to use only a center word, and have our model predict all of the surrounding context words?
This is a common training technique for embeddings as well, and is known as the Skip-Gram model.

\subsection{Spacy Example of Word Embeddings}

\begin{python}
  import spacy

  sample_words = ["epilepsy", "seizure", "patient", "language"]

  nlp = spacy.load("en_core_web_md")
  docs = [nlp(sample) for sample in sample_words]

  #Spacy's Medium English Model Stores Words As Vectors of Dimension 300
  #Note each word has this vector size
  for doc in docs:
      assert doc.vector.shape[0] == 300

  #The cosine distance, ranging from [0,1], of the vectors tells us how similar the words are
  print(docs[0].similarity(docs[1])) #epilepsy and seizure -> 0.9999
  print(docs[0].similarity(docs[2])) #epilepsy and patient -> 0.4434
  print(docs[0].similarity(docs[3])) #epilepsy and language -> 0.1273
\end{python}

\subsection{Character and Subword Embeddings}

In Example 1, we covered the issue of missing vocabulary that is frequently encountered when using word-level tokenization.
However, there are other strategies that are commonly used that help to avoid this issue.
One such simple way around out-of-vocabulary tokens is to use character-level tokenization.
Instead of having a vocabulary that is tens of thousands of words long, you only have a vocabulary that contains the characters in
your language, with the addition of punctuation, digits, and a few special tokens for sentence and word marking.
As long as the tokenizer is able to mark which characters start a word and which characters are continuations of a word, then the original sequence of words can always be recovered.

Unfortunately, the tradeoff for a small vocabulary size with no unknown word entries is that the length of tokens to represent a sentence is clearly much larger.
Empirically, these models do not perform as well, partially because of this sequence length but also because of the loss of information that is represented in word vectors.

In practice, the best models are often somewhere in the middle, in what is known as word-piece, or subword, vocabularies.
These tokenizers and vocabularies contain a rich set of common words, but also have subword units that can be used to piece together out-of-vocabulary words.
These subword units behave like normal word vectors and can store semantic and syntactic information.


\begin{python}
  # Example 3 - Wordpiece Tokenizers

  from transformers import BertTokenizer

  sample = "After a temporal lobe resection, the " \
           "atonic and clonic seizure frequency fell " \
           "by 50%."

  bert_tokenizer = BertTokenizer.from_pretrained(
      'bert-base-uncased'
  )

  encoded_ids = bert_tokenizer.encode(sample)
  encoded_tokens = bert_tokenizer.convert_ids_to_tokens(
      encoded_ids
  )

  for token, id  in zip(encoded_tokens, encoded_ids):
      print("{}: {}".format(token,id))
\end{python}

In example two,
% \subsubsection{Vocabularies and Medical Embeddings}

\section{NLP Tasks in Epilepsy}

Now that we have an understanding of some of the basics of computational representations of language data, let us take a general look at some of the modern use cases for NLP in industry, 
catered toward epileptologists and other medical practitioners. In this section We will briefly cover examples of that all follow a common pattern. Once we establish this pattern, the
following sections will take a deeper look with working examples of Financial Impact Analysis and SUDEP prediction.

\subsection{Unstructured Text - Retrospective Research}
Within the medical field, there are many sources of unstructured text that are primarily aimed as a either a communication channel between specialists, or a way for patients to pass information that isn't captured in a structured format.
Clinical Notes, Progress Notes, Operative Notes, Discharge Summaries, Pathology Reports, Surgery Reports; all of these sources carry information, but traditionally require experts in the field to extract it.
If we can teach systems to read through this unstructured text and perform respectably close to these experts, we can perform large-scale clinical retrospective research much faster and at a much lower cost.

Because these pre-trained language models can draw upon information from so many domains and have a sound understanding of language, they need relatively few
labeled examples to begin to perform tasks such as classifying text. All they have to learn is the task at hand, not the nuances of languages.
If a team of experts is able to identify a set of labels in which they are interested, they can label this data in a single shot. The next step is to fine-tune a pre-trained language model, often by just attaching a logistic regression layer to the outputs of the language model,
to recognize each of these labels.

Instead of having to wait for, not to mention pay for, experts to do large-scale retrospective research, fine-tuned pre-trained language models can perform admirably on text classification with, in many cases, only a few hundred labeled examples.
The trained model can then read through and extract targeted text much faster and much cheaper.
Studies have already been done applying this strategy, using three pre-trained neural language models, \texttt{Bio\textunderscore ClinicalBERT}, to extract
seizure frequency from epilepsy clinical notes.\cite{10.1093/jamia/ocac018}

\subsection{Unstructured Text - Quality of Life}
When treating patients with epilepsy, often our principal aim is to target improving quality of life. Seizure frequency is a numeric represenation that has a measurement
that is well-understood and directly impacts quality of life. However, many measurements of quality of life come directly from patient surveys and other more subjective
metrics that can be more difficult to interpret at scale. While often these surveys call for responses on numeric scales, many offer unstructured text as a method for
patients to freely communicate other information relevant to quality of life, or further explain their responses. This information is of exceptional importance, but it
can be difficult to parse through it at scale without modern natural language processing techniques.

Sentiment analysis, the task of extracting and studying subjective sentiment in text and speech data, is yet another field of NLP that has greatly benefited from the
widespread adoption of transformer architectures, particularly in specialized domains. Freeform responses that can be evaluated as positive or negative, or sometimes into more
granular sentiments, can be learned with machine learning methods. It's worth noting that naive methods on sentiment tend to capture some cases well. The words love and hate are
strong signals, but even their simple negation becomes a diluted signal with possible interpretations of sarcasm that become subjective. Add in the many colorful ways that people
write, and we start to see why rule-based text-mining systems for sentiment are so difficult.

Sentiment analysis models using transformer architectures such as RoBERTa aim to project the large-dimensional output vector of the text, and project this semantic representation
into a basic sentiment of two class, positive or negative. Very often a third class is used to represent neutral sentiment. This method should be used when analyzing free-form responses
in patient surveys using NLP models trained on sentiment. While a model trained on medical data is ideal, it's quite impressive how well a model trained on movie reviews will transfer
over to other sentiment tasks.

The same methods for analyzing basic sentiment can be used to label tones, more granular emotional classifications, or other task-specific points of interest.
It is important to have the survey designers, and/or other domain experts, label these targeted semantic
expressions of their patients. Once a model is trained with a set of positive examples of these targets, it can be used on any text response.

The datasets provided by SeizureTracker, as part of their It's Not Just Seizures (INJS) initiative, provides an excellent example of unstructured text responses being
used as a guide to a suggested set of useful labels that can be learned with NLP techniques.

\subsection{Named Entity Recognition and PII}

Named Entity Recognition (NER) is a subtask of information extraction that classifies unstructured text into personal names, locations, time indications, organizations, and other tags.
NER is an important problem in medicine, if nothing else to provide an ability to mask out Personally-Identifiable Information.
If a model, probably a transformer architecture in practice, is able to identify text as a first name, a home address, or a credit card number, then we can anonymize this information to mask information that is PII-specific.

It should be noted that, although some of this text is easy to catch with rule-based approaches and regular expressions, such as a credit card number, other information requires contextual understanding and an ML solution
to achieve the necessary performance.

Spacy, NLTK, and Stanford NLP all provide out-of-the-box NER models that perform well on unstructured text. Let's look at a Spacy example in practice.

\begin{python}
  # Example 4 - Simple PII Anonymizer

  import spacy

  sample = "The patient, John Smith, began feeling " \
           "symptoms at his home in Seattle, and was " \
           "taken to Virginia Mason Medical Center " \
           "for treatment"
  nlp = spacy.load("en_core_web_md")

  doc = nlp(sample)

  for word in doc.ents:
      sample = sample.replace(word.text, word.label_)

  print(sample)
  # The patient, PERSON, began feeling symptoms at his
  # home in GPE, and was taken to ORG for treatment

\end{python}

\subsection{Clinical Trial Matching and Surgical Candidacy}

There has been an emergence of NLP technologies aimed at clinical trial matching. Different patient characteristics provide against eligibility
criteria of available trials, and because much of this data comes in the form of unstructured text, NLP techniques can be used to turn target
creating structures features for clinical trial matching.

Criteria2Query \cite{10.1093/jamia/ocy178} is a hybrid system that uses NLP methods to transform text into eligibility criteria features that can
be extracted with structured queries. The Clinical Trial Parser and accompanying dataset of 3,314 trials\cite{tseo2020information}, made
publicly available from Meta Research, is a system with a similar goal that combines a number of information extraction techniques from NLP to determine
trial level eligibility.

Surgical Candidacy scoring models are yet another example of a classification task wherein recent NLP approaches have been used and evaluated \cite{Wissel2019ProspectiveVO}
in epilepsy research. While this is a very broad introduction into some of the ways that the field has increasingly incorporated computational linguistics in recent years, it is enough
to recognize a common pattern. As a general rule, when evaluating the potential candidacy of an NLP model to your specific task, ask yourself if the following criteria hold true:
\begin{enumerate}
  \item Is there text data in this task that provides signal toward a targeted classification that typically requires human involvement?
  \item Have others in the medical field tried using NLP models on a similar task?
  \item Does the task that requires either substantial amount of time, money, or scarce resources of domain expertise?
  \item Is there a dataset I can use that is in good shape and does my group have full permissions to work with it on the required hardware?
\end{enumerate}

If the answer is yes to any of these questions, it may be worth trying some of these off-the-shelf techniques or even reaching out to a data scientist or machine learning engineer.
If the answer is yes to all four, then it is definitely worth your time.

\section{Full Walkthrough - Brain Surgery and Financial Impact of Epilepsy}

In this section we will demonstrate how the techniques discussed above can be applied to a practical example within the field of Epilepsy.
Modern NLP techniques are a powerful tool to analyze large amounts of unstructred or semi structured text data, and we can employ them to develop a coding scheme and significantly reduce the time required to apply it to new data.

We are providing self contained code examples in the acompanying GitHub repository\footnote{https://github.com/chris-boson/epilepsy} that makes use of a general purpose NLP library\footnote{https://github.com/robmsylvester/sheepy} we have developed to make current industry standard tools and libraries more accessible to researchers and practitioners.


\subsection{Data Analysis}
For this section we started out with fairly small dataset ($\sim150$ rows) derived from survey responses regarding  impact and treatment of epilepsy\footnote{Seizure Tracker}:
\begin{displayquote}
    Do you have any comments on the financial impact of epilepsy?

    Do you have any comments about brain surgery in general?
\end{displayquote}
This dataset is small enough to manually inspect, but we the techniques discussed here scale well beyond.
\subsubsection{Embeddings}

\subsubsection{Dimensionality Reduction and Clustering}
\subsubsection{Cluster Labeling}
\subsubsection{Annotation}
We can use the cluster labels as a starting point to decide on a coding scheme. For the general comments we decided on the following labels:
\begin{displayquote}
    \texttt{"Not eligible"},
    \texttt{"Last resort"},
    \texttt{"Would never do it"},

    \texttt{"Considering it"},
    \texttt{"Was Unsuccessful"},
    \texttt{"Was partially successful"},

    \texttt{"Was successful"},
    \texttt{"Side effects"},
    \texttt{"Risk"},

    \texttt{"Too expensive"},
    \texttt{"Complications"},
    \texttt{"Unknown outcome"},

    \texttt{"Unnecessary"},
    \texttt{"Cannot find origin"}
\end{displayquote}
We treat this as a so called multilabel classification problem, where a given sample can have multiple true labels simultaneously (i.e. a surgery can be successful and have side effects).

The labeling scheme should cover all the aspects of the data of interest.
It also needs to be unambiguous, meaning that two experts independently generating the annotations should largely agree on what the correct labels are for each sample.
If human experts cannot agree, the model will most likely perform poorly as well, as it gets inconsistent signals during training.
If this is the case, the coding scheme or annotation instructions should be revised.
We will discuss in Section \ref{metrics_and_sanity_checking} how to go about analyzing model performance and uncover issues in data and annotations.



%     - SHAP
%     - Analyze mistakes
%     - Inference


\section{Generative Models and Large Language Models}

A recent trend, one that has gained notoriety in circles far outside of machine learning, is the usage of generative AI
and large language models \textit{LLMs} for both ML and non-ML applications. These are largely unsupervised, or sometimes
semi-supervised, algorithms that generate content from analyzing existing content. This content is often text, but in recent
multi-modal models such as OpenAI's GPT-4, can be other modalities such as images as well, or  even a combination thereof.

Large Language Models have gained notoriety specifically because they have arguably reached human-level performance, and according
to some even passed it, on this task of generating text from a prompt (often called context). The word "large" here is an understatement,
as these models have billions upon billions of parameters, and rely on neural architectures that stack transformers and other large
computation operations across many GPU's.

What makes these newer language models such as OpenAI's \textit{ChatGPT}, Meta AI's \textit{Llama}, or Google's \textit{Bard} so special
is their ability to generate high-quality content at scale without a loss in accuracy. In the case of ChatGPT, a fine-tuning process with
human labelers called Reinforcement Learning With Human Feedback \textit{RLHF} targeted iteratively improving the model by specifically
rewarding high-quality content and penalizing toxic feedback. Because these models are trained on text across the internet, which
unsurprisingly included troves of nonsense and hate speech, these models must be pushed toward not pure unsupervised generation of content,
as this can be dangerous.

\subsection{LLMs and Epilepsy}

Consider the following example from OpenAI's ChatGPT (March 14, 2023 version)

\begin{python}
  prompt_1 = "After a temporal lobe resection, why would we expect satonic eizure frequency to fall in a patient?"

  #... send to ChatGPT over OpenAI Api ...

  response = "After a temporal lobe resection, we may expect atonic seuzyre frequency to decrease in a patient if \
    the atonic seizures were originating from or spreading to the area of the brain that was removed during the \
    surgery. \
        Atonic seizures, also known as drop attacks or drop seizures, are a type of seizure characterized by a sudden \
    loss of muscle tone, which can cause the person to fall to the ground or slump over. Atonic seizures are often \
    associated with epilepsy and can be difficult to control with medication. \
        If atonic seizures are localized to the temporal lobe or are spreading from that area, temporal loe resection \
    can be an effective treatment option. By removingg the portion of the temporal lobe where the seizures are originating \
    or spreading from, the surgery can reduce or eliminate atonic seizures. \
        However, it is worth noting that atonic seizures may also be caused by other factors, such as structural abnormalities \
    in other parts of the brain, and in these cases, temporal lobe resection may not be effective. Additionally, the outcome \
    of the surgery may vary depending on various factors, such as the extent of the surgery, the location of the seizure focus \
    and the patient's individual characteristics. Therefore, it is essential to evaluate each patient's individual situation \
    to determine whether temporal lobe resection is an appropriate treatment option for them."
\end{python}

Considering this is a generated response, it goes without saying that this is a promising step toward advancements in tools
in the medical field. The models are so large that they can fit in a rich back-and-forth discussion, without a sacrifice
in depth.

\begin{python}
    prompt_1 = "In such cases if a multi focal seizure is present, when would nerve stimulation be preferred over surgery"
  
    #... send to ChatGPT over OpenAI Api ...
  
    response = COPY THIS RESPONSE IN
  \end{python}

\subsection{Prompt Design, Context, and Fine-Tuning}

\subsection{Revisiting Our Tasks}

\subsection{Notes on The Future of Generative AI in Medicine}
