\section{Full Walkthrough - Brain Surgery and Financial Impact of Epilepsy}

In this section we will demonstrate how the techniques discussed above can be applied to a practical example within the field of Epilepsy.
Modern NLP techniques are a powerful tool to analyze large amounts of unstructred or semi structured text data, and we can employ them to develop a coding scheme and significantly reduce the time required to apply it to new data.

We are providing self contained code examples in the acompanying GitHub repository\footnote{https://github.com/chris-boson/epilepsy} that makes use of a general purpose NLP library\footnote{https://github.com/robmsylvester/sheepy} we have developed to make current industry standard tools and libraries more accessible to researchers and practitioners.


\subsection{Data Analysis}
For this section we started out with fairly small dataset ($\sim150$ rows) derived from survey responses regarding  impact and treatment of epilepsy\footnote{Seizure Tracker}:
\begin{displayquote}
    Do you have any comments on the financial impact of epilepsy?

    Do you have any comments about brain surgery in general?
\end{displayquote}
This dataset is small enough to manually inspect, but we the techniques discussed here scale well beyond.
\subsubsection{Embeddings}

\subsubsection{Dimensionality Reduction and Clustering}
\subsubsection{Cluster Labeling}
\subsubsection{Annotation}
We can use the cluster labels as a starting point to decide on a coding scheme. For the general comments we decided on the following labels:
\begin{itemize}
    \item Not eligible
    \item Last resort
    \item Would never do it
    \item Considering it
    \item Was Unsuccessful
    \item Was partially successful
    \item Was successful
    \item Side effects
    \item Risk
    \item Too expensive
    \item Complications
    \item Unknown outcome
    \item Unnecessary
    \item Cannot find origin
\end{itemize}

% - Large scale data analysis
%   - Embeddings
%   - Dimensionality reduction and clustering
%   - Cluster labeling
%   - Annotation
% - Research process
%   - Sheepy
%   - Reproducible results
%     - Experiment tracking
%     - Parameter tuning
%   - Metrics and sanity checking
%     - SHAP
%     - Analyze mistakes
%     - Inference

  % walk through the full process of developing a model to predict the financial impact of epilepsy, and how to do this in a way that is both accurate and scalable.
