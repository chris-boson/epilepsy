\section{Introduction}
In this chapter, we will present the essential skills and tools for effective collaboration, version control, and programming. Whether you are new to these concepts or an experienced practitioner, this chapter will serve as a comprehensive guide to Github, version control systems, Integrated Development Environments (IDEs) and two widely used programming languages for statistical analysis and machine learning - R and Python.

All too often in medicine and academia we write scripts that may or may not run on other computers. We have results that can be difficult to replicate or visualize, and we often spend too much time writing code for something that some expert has already spent much more time on. By introducing the fundamentals of version control and Github, the readers will be able to track and manage their projects and any changes therein with ease. As a result, we expect practitioners to have code that is reproducible by others.

The chapter will cover essential programming concepts in R and Python. Using this knowledge the reader will be presented with a practical example that illustrates an end-to-end development cycle for statistical analysis. Throughout the chapter, reproducible and clean code will be emphasized. Additionally, best practices and benefits of each language, data structures, how to read in and join data, data cleaning, wrangling and visualizations will be presented. These concepts will be introduced via a practical example using simulated seizure counts and spikes data. We hope that by following along with our hands-on example, the readers will gain valuable insights into the iterative process of data analysis and visualization and their ability to help uncover meaningful patterns and insights from the data.
