\section{Introduction}
<<<<<<< HEAD
Running code to solve problems in the domain of epilepsy is very much like running code anywhere within the medical domain.
Taken a level higher, code in the medical field should always follow the practices of proper engineering work, and always follow the scientific method.
As practitioners of medicine and as engineers it is critical that we focus our energy on what makes our experimental code unique, and rely on standards and libraries for the rest.

All too often in medicine and academia we write scripts that may or may not run on other's computers.
We have results that can be difficult to replicate or visualize, and we often spend too much time writing code for something that some expert has already spent much more time on.
Before we ever write code unique to epilepsy, we should do the following:
\begin{enumerate}
    \item Use A Proper IDE
    \item Use Version Control
    \item Use Virtual Environments
\end{enumerate}

The end result of all of this is that you know, for a fact, that somebody else can download your code, click one button (or complete a small number of simple steps), and replicate or verify your results.
Conversely, if your peers do the same for their projects, you can verify and build on their work.
This is true for code in Python, R, Matlab, Java, C++, or whatever language(s) you choose.
This is true for machines running Linux, Windows, or MacOS, and on many types of systems.

In this chapter, we will outline the basics of getting started with a proper development environment utilizing Git.
We will cover using Python and R at a depth that covers the basics of the language, using virtual environments, and examples of statistical analysis.
By the end of this chapter, you should have the tools necessary to be able to download, run, and modify code covered in this textbook.
You will also be equipped to start running your own experiments in a proper fashion so that they can follow these best practices.

\subsection{Data Sharing for ML (move somewhere?)}
=======
In this chapter, we will present the essential skills and tools for effective collaboration, version control, and programming. Whether you are new to these concepts or an experienced practitioner, this chapter will serve as a comprehensive guide to Github, version control systems, Integrated Development Environments (IDEs) and two widely used programming languages for statistical analysis and machine learning - R and Python.

All too often in medicine and academia we write scripts that may or may not run on other computers. We have results that can be difficult to replicate or visualize, and we often spend too much time writing code for something that some expert has already spent much more time on. By introducing the fundamentals of version control and Github, the readers will be able to track and manage their projects and any changes therein with ease. As a result, we expect practitioners to have code that is reproducible by others.

The chapter will cover essential programming concepts in R and Python. Using this knowledge the reader will be presented with a practical example that illustrates an end-to-end development cycle for statistical analysis. Throughout the chapter, reproducible and clean code will be emphasized. Additionally, best practices and benefits of each language, data structures, how to read in and join data, data cleaning, wrangling and visualizations will be presented. These concepts will be introduced via a practical example using simulated seizure counts and spikes data. We hope that by following along with our hands-on example, the readers will gain valuable insights into the iterative process of data analysis and visualization and their ability to help uncover meaningful patterns and insights from the data.
>>>>>>> 0e8375a58a88c1ccca571510318da256d03c3d60
