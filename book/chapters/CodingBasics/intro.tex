\section{Introduction}
Running code to solve problems in the domain of epilepsy is very much like running code anywhere within the medical domain.
Taken a level higher, code in the medical field should always follow the practices of proper engineering work, and always follow the scientific method.
As practitioners of medicine and as engineers it is critical that we focus our energy on what makes our experimental code unique, and rely on standards and libraries for the rest.

All too often in medicine and academia we write scripts that may or may not run on other's computers.
We have results that can be difficult to replicate or visualize, and we often spend too much time writing code for something that some expert has already spent much more time on.
Before we ever write code unique to epilepsy, we should do the following:
\begin{enumerate}
    \item Use A Proper IDE
    \item Use Version Control
    \item Use Virtual Environments
\end{enumerate}

The end result of all of this is that you know, for a fact, that somebody else can download your code, click one button (or complete a small number of simple steps), and replicate or verify your results.
Conversely, if your peers do the same for their projects, you can verify and build on their work.
This is true for code in Python, R, Matlab, Java, C++, or whatever language(s) you choose.
This is true for machines running Linux, Windows, or MacOS, and on many types of systems.

In this chapter, we will outline the basics of getting started with a proper development environment utilizing Git.
We will cover using Python and R at a depth that covers the basics of the language, using virtual environments, and examples of statistical analysis.
By the end of this chapter, you should have the tools necessary to be able to download, run, and modify code covered in this textbook.
You will also be equipped to start running your own experiments in a proper fashion so that they can follow these best practices.